\section{Related Work}

\subsection{Overview of Existing Network Security Analysis Techniques}
\begin{itemize}
    \item Begin by introducing traditional network security methods, such as basic intrusion detection systems and firewalls.
    \item Discuss the limitations of these methods, especially in handling advanced persistent threats (APTs) and distributed denial of service (DDoS) attacks.
\end{itemize}


\subsection{Multilayer and Multidimensional Security Analysis}
\begin{itemize}
    \item Discuss the advanced concepts of multilayer security architectures, citing key studies and practical cases.
    \item Mention specific tools and techniques that utilize machine learning and big data analytics, such as IBM's QRadar or Splunk.
\end{itemize}

\subsection{Artificial Intelligence in Network Security}
\begin{itemize}
    \item Provide detailed information on how artificial intelligence, particularly machine learning and deep learning technologies, are enhancing the capabilities of network security analysis.
    \item Illustrate how AI technologies improve anomaly detection, behavioral analysis, and event correlation analysis.
\end{itemize}

\subsection{Multisource Data Integration and Analysis Techniques}
\begin{itemize}
    \item Explore how data from different sources (such as network traffic, event logs, and system logs) is integrated to provide a comprehensive security perspective.
    \item Reference innovative research in this area, potentially involving data fusion techniques and heterogeneous data analysis.
\end{itemize}

\subsection{Dynamic Threat Graphs and Visualization Techniques}
\begin{itemize}
    \item Introduce how other researchers use threat graphs to represent the security state of network environments.
    \item Discuss how these techniques help security teams quickly identify threats and understand patterns of attack behaviors.
\end{itemize}
