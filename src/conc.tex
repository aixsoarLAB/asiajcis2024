\section{Conclusion}\label{sec:conclusion}
This paper has presented a comprehensive intelligent detection tool that leverages advanced artificial intelligence techniques to enhance network security through a multi-layer threat graph approach. Our experiments demonstrate that this tool significantly improves the detection of sophisticated and multi-stage cyber threats, outperforming traditional security systems in both synthetic and real-world environments.

\subsection{Key Findings}\label{sec:key-findings}
The key findings from our research include:
\begin{itemize}
    \item The integration of machine learning and deep learning algorithms enables our tool to effectively identify both known and novel threats by analyzing patterns in network traffic, system logs, and intrusion detection outputs.
    \item Our experimental results highlight the tool's enhanced capability in precision and recall, particularly in detecting encrypted and anomalous traffic, which are common vectors for advanced persistent threats (APTs).
    \item The visualization capabilities of the multi-layer threat graph provide intuitive and actionable insights, allowing security teams to swiftly respond to potential threats and understand the broader security landscape.
\end{itemize}

\subsection{Implications for Cybersecurity}\label{sec:implications}
The development of this tool signifies a substantial advancement in cybersecurity practices. By providing a deeper and more nuanced understanding of threat dynamics, our tool supports a proactive security posture, enabling organizations to preemptively address vulnerabilities and mitigate potential attacks more effectively.

\subsection{Future Research Directions}\label{sec:future-research}
Future research will focus on several key areas to further enhance the detection tool's capabilities:
\begin{itemize}
    \item \textbf{Real-Time Data Processing:} Improving the tool's ability to process and analyze data in real-time will help in quicker threat identification and response.
    \item \textbf{Integration of More Data Sources:} Expanding the types of data inputs to include more varied network and endpoint data will enhance the depth and accuracy of threat detection.
    \item \textbf{Automated Response Mechanisms:} Developing automated response features that can not only detect but also respond to security threats in an automated fashion will be a significant step forward.
    \item \textbf{Cross-Platform Compatibility:} Ensuring the tool is compatible across different platforms and environments will broaden its applicability and utility.
\end{itemize}

In conclusion, the intelligent detection tool introduced in this paper provides a vital asset in the field of cybersecurity, offering enhanced detection capabilities and a robust framework for future enhancements. As cyber threats continue to evolve, so too must our approaches to detecting and combating them. Our future work will continue to build on this foundation with the aim of developing a universally applicable, highly adaptive security solution.
