\section{System Architecture}

\begin{figure*}[!htb]
    \centering
    \includegraphics[width=0.75\textwidth]{../images/Screenshot 2024-05-03 123930.png}
    \caption{Overview of the System Architecture}
    \label{fig:systemarch}
\end{figure*}



\subsection{Model 1: Anomaly Detection and Interpretation}
The first model focuses on evaluating event logs and NIDS alert logs. This includes several components designed to parse, interpret, and extract meaningful patterns from the data.

\subsubsection{Interpreter}
The Interpreter component processes incoming data by converting unstructured or semi-structured information into a structured format that can be analyzed for anomalies.

\subsubsection{Behavior Extractor}
This component extracts behavioral patterns from the data provided by hosts, particularly from event logs, identifying deviations from normal operations that signal potential security threats.

\subsubsection{Context Builder}
Integrates contextual information into the data interpretation process, enhancing the accuracy and relevance of the anomaly detection.

\subsection{Model 2: Tagging and Network Analysis}
This model evaluates NIDS alert logs and network flow pcap files. It structures these data sources and utilizes the information to build a comprehensive network graph, which is crucial for visualizing network interactions and identifying potential threats.

\subsubsection{Data Structuring}
Organizes raw data into a usable format, integrating various data sources, including NIDS alert logs and pcap files, essential for the initial detection of anomalies.

\subsubsection{Abnormal Tagging}
Following data structuring, this process classifies and labels data instances showing anomalous characteristics, aiding in the focused analysis of potential threats.

\subsubsection{Network Graph}
Constructs a dynamic representation of network interactions, fundamentally supporting the visualization of threat dispersion within the network.

\subsection{Model 3: Traffic and Feature Analysis}
This final model evaluates pcap files exclusively, focusing on the detection of malicious traffic and the extraction of relevant features for a deeper analysis of anomalies.

\subsubsection{Malicious Traffic Detection}
Employs algorithms to identify and highlight traffic patterns that correspond to previously identified malicious behaviors, enhancing the system's preventive capabilities.

\subsubsection{Graphical Anomaly Interaction Analysis}
Analyzes the network graph to detect and examine anomalous interactions between network nodes, providing intuitive insights into the nature and potential impact of detected threats.

\subsubsection{Feature Extractor}
Isolates and extracts relevant features from network flow data, crucial for the precise identification and classification of network anomalies.

\section{Conclusion}
The structured approach of dividing the system into three distinct but interrelated models allows for a layered defense against cyber threats. By systematically structuring and analyzing data through these models, our system enhances the capability to detect, visualize, and respond to sophisticated cyber threats effectively.
